% Document class and font size
\documentclass[a4paper,10pt]{extarticle}

% Packages
\usepackage[utf8]{inputenc} % For input encoding
\usepackage{geometry} % For page margins
\geometry{letterpaper, margin=0.75in} % Set paper size and margins
\usepackage{titlesec} % For section title formatting
\usepackage{enumitem} % For itemized list formatting
\usepackage{hyperref} % For hyperlinks
\usepackage{tabularx}
\usepackage{fancyhdr}
\usepackage{datetime}
\usepackage{multicol}

% Formatting
\setlist{noitemsep} % Removes item separation
\titleformat{\section}{\large\bfseries}{\thesection}{1em}{}[\titlerule] % Section title format
\titlespacing*{\section}{0pt}{\baselineskip}{\baselineskip} % Section title spacing
\setlength{\parindent}{0pt} % Removes indent
\newdateformat{monthyeardate}{% Date formatting for header
  \monthname[\THEMONTH] \THEYEAR}
%%%%%%%%%%

% Begin document
\begin{document}

% Disable page numbers
\pagestyle{fancy}
\renewcommand{\headrulewidth}{0pt}
\fancyhead{}
\fancyhead[L]{\textit{Kyle Wong}}
\fancyhead[R]{\textit{\monthyeardate\today}}
\thispagestyle{empty} % Remove header from the first page

% Header
\begin{flushleft}
\textbf{\LARGE Kyle Wong}\\[0.5em] % Name
Recent graduate from Sidney Sussex College, University of Cambridge
\\ Email: \href{mailto:kyleyhw@gmail.com}{kyleyhw@gmail.com} % Contact info
\\ GitHub: \url{https://github.com/kyleyhw}

\vspace{1em}
\textit{A highly motivated recent Master's graduate from the University of Cambridge with a strong background in Astrophysics, Mathematics, and computational science. Experienced in cosmological research, gravitational wave data analysis, and radio astronomy. Seeking a challenging research or data science role to apply expertise in Python, scientific computing, and data analysis.}
\end{flushleft}

% Education Section
\section*{EDUCATION}
\textbf{Sidney Sussex College, University of Cambridge}, Cambridge, England, United Kingdom \hfill 2024 | 2025\\
MASt/Part III in Astrophysics\\
Graduated with equiv. Good Honors\\



\textbf{Victoria College, University of Toronto}, Toronto, Ontario, Canada \hfill 2020 | 2024\\ % University name and location
Honours Bachelor of Science (HBSc) in Physics \& Mathematics\\
Graduated with High Distinction\\


\textbf{German Swiss International School}, Hong Kong SAR, China \hfill 2010 | 2020\\ % University name and location
International Baccalaureate (IB) Diploma in 6 subjects\\
International Advanced Subsidiary (AS) level in 1 subject\\
International General Certificate of Secondary Education (IGCSE) in 12 subjects

% Experience Section
\section*{RESEARCH EXPERIENCE}
\textbf{Master's of Advanced Study Research Project} \hfill Cambridge, England, United Kingdom\\
\textit{Postgraduate Student} \hfill October 2024 | September 2025\\
Supervised by Prof. Anastasia Fialkov at the Institute of Astronomy, University of Cambridge
\begin{itemize}
    \item Research topic: implementation and statistical testing of variable initial conditions (cosmologies) and variable resolution in 21cmSPACE code package for simulating the distribution of hydrogen gas clouds from in the early universe, making possible efficient forecasting of future radio astronomy experiments, most notably for the Square Kilometre Array (SKA)
    
    \item Relevant skills: simulation, MATLAB, high performance computing, data visualization, cosmology
\end{itemize}

\textbf{Canadian Institute for Theoretical Astrophysics\\
    Summer Undergraduate Research Fellowship} \hfill Toronto, Ontario, Canada\\
\textit{Undergraduate Researcher} \hfill May 2023 | December 2023\\
Supervised by Dr. Philippe Landry at the Canadian Institute for Theoretical Astrophysics
\begin{itemize}
    \item Research topic: probing neutron star tidal deformability from gravitational wave signals using Markov chain Monte Carlo (MCMC) parameter estimation, and incorporating new models for neutron star equation of state correlations in the analysis pipeline of Laser Interferometer Gravitational-Wave Observatory (LIGO) Scientific Collaboration gravitational wave data

    \item Relevant skills: simulation, MCMC, Bayesian inference, high performance computing
\end{itemize}

 
\textbf{McGill Space Institute Summer Undergraduate Research Award} \hfill Montreal, Quebec, Canada\\
\textit{Undergraduate Researcher} \hfill May 2022 | April 2023\\
Supervised by Prof. Adrian Liu at the Trottier Space Institute (formerly McGill Space Institute) of McGill University
\begin{itemize}
    \item Research topic: incorporating statistical priors into the power spectrum data estimator used in the data pipeline, for analysis of radio astronomy data from the Hydrogen Epoch of Reionization Array (HERA) collaboration's cosmic dawn experiment, using the Python programming language

    \item Relevant skills: simulation, Fourier transform, radio astronomy, radio frequency interference
\end{itemize}


\newpage
\section*{HONOURS, AWARDS AND SCHOLARSHIPS}
\begin{itemize}
    \item Dean's List Scholar in the Faculty of Arts \& Science \hfill 2022, 2023, 2024

    \item Canadian Institute for Theoretical Astrophysics \\
    Summer Undergraduate Research Fellowship (CITA SURF) \hfill 9,500 CAD, 2023
    
    \item Birkenshaw Family Scholarship II \hfill 1,000 CAD, 2023

    \item McGill Space Institute (now Trottier Space Institute) \\Summer Undergraduate Research Award (MSI SURA) \hfill 7,000 CAD, 2022

    \item David and Louise Fraser Scholarship \hfill 2,500 CAD, 2022

    \item University of Toronto Scholar \hfill 1,500 CAD, 2022

    \item Birkenshaw Family Scholarship \hfill 1,000 CAD, 2022

    \item Received offer for University of Toronto \\Natalia Krasnopolskaia Memorial Summer Undergraduate Research Fellowship \\(declined due to commitment with MSI SURA) \hfill 2022
\end{itemize}

% Skills Section
\section*{SKILLS}
\begin{itemize}
    \item \textbf{Programming:} Proficient in Python for scientific computing and data analysis, with additional experience in MATLAB and Java.
    \item \textbf{Data Analysis:} multi-dimensional data visualization, fast Fourier transform, numerical methods (differentiation, integration, root finding, ODE/PDE solutions), Monte-Carlo methods, and symbolic computing.
    \item \textbf{Software \& Libraries:} Bilby (gravitational wave parameter estimation), 21cmSPACE, CAMB, recfast++ (cosmological simulation).
    \item \textbf{Machine Learning (ML):} Solid understanding of fundamental machine learning techniques, including the MCMC method and Convolutional Neural Network (CNN) architectures. Experience in implementing ML models from first principles.
    \item \textbf{AI Tools:} Google Gemini, Gemini CLI, and ChatGPT for academic and scientific research and coding.
\end{itemize}

\section*{PRESENTATIONS}
\begin{itemize}
    \item Interim progress presentation, at the Cambridge Cosmic Dawn Group \hfill 2024

    \item \textit{Estimating Neutron Star Tidal Deformability}, at the CITA Undergraduate Research Showcase \hfill 2023

    \item PHY478 Physics Project final presentation \hfill 2023

    \item Two presentations given at the CITA Compact Objects Group, as part of PHY478 Physics Project \hfill 2023
\end{itemize}

\section*{FEATURED CODE REPOSITORIES}

\textbf{sound\_simulation} (\url{https://github.com/kyleyhw/sound_simulation})
\begin{itemize}
    \item A numerical simulation of physical waves in N-dimensions, implemented in Python. The simulation allows for the creation of physical boundaries and obstacles, and models their interaction with the waves.
    \item Future work aims to use this simulation to generate training data for machine learning models.
    \item \textit{Skills: Python, NumPy, Matplotlib, numerical modeling, physics simulation.}
\end{itemize}

\vspace{0.5em}

\textbf{digit\_recognition} (\url{https://github.com/kyleyhw/digit_recognition})
\begin{itemize}
    \item A project to build a convolutional neural network (CNN) from scratch in Python for handwritten digit recognition. This was undertaken to develop a fundamental understanding of machine learning principles without relying on high-level libraries like PyTorch or TensorFlow.
    \item \textit{Skills: Python, NumPy, machine learning, neural networks, computer vision.}
\end{itemize}

\vspace{0.5em}

\textbf{driver\_assignment} (\url{https://github.com/kyleyhw/driver_assignment})
\begin{itemize}
    \item An optimization tool to solve the problem of assigning passengers to drivers for group trips. The tool uses an algorithm to find an optimal assignment based on driver and passenger constraints.
    \item \textit{Skills: Python, optimization, algorithms, problem-solving.}
\end{itemize}
\newpage

\section*{SELECTED COURSES}
\begin{multicols}{2}
\begin{itemize}
    \item Gravitational Waves \& Numerical Relativity \\(Cambridge, 2025)
    \item Astrostatistics (Cambridge, 2025)
    \item Cosmology (Cambridge, 2024)
    \item General Relativity (Cambridge, 2024)
    \item General Relativity (Toronto, 2024)
    \item Relativity Theory II (Toronto, 2024)
    \item Relativity Theory I (Toronto, 2023)
    \item Physics Project (Toronto, 2023)
    \item Computational Physics (Toronto, 2023)
    \item Computational Astrophysics (Toronto, 2023)
    \item Advanced Classical Mechanics (Toronto, 2023)
    \item Geometry of Curves and Surfaces (Toronto, 2023)
\end{itemize}
\end{multicols}

\section*{RESEARCH INTERESTS}
\begin{itemize}
    \item Computational data analysis

    \item Simulation programming

    \item Fundamental machine learning, artificial intelligence
    
    \item Astrophysical phenomena (including gravitation, cosmology, black holes \& compact objects, dark matter, dark energy)
\end{itemize}


% Experiences Section
\section*{RELEVANT EXPERIENCE}
\begin{itemize}
    \item \textbf{Teaching:} Private Tutor for physics & mathematics (IGCSE, A-levels, IB syllabi). \hfill 2020 – 2022, 2024 – 2025
    \item \textbf{Professional Development:} Attended a one-week intensive bootcamp on astrophysics at the University of Pennsylvania (HERA Collaboration). \hfill 2022
    \item \textbf{Volunteering:}
    \begin{itemize}
        \item Involved in public astronomical observing nights at the Institute of Astronomy.
        \item Involved in Beaver Scouts outreach visits at the Institute of Astronomy.
        \item Volunteered for RiseWise HK, serving as an assistant soccer coach for children with special educational needs.
        \item Travelled to Chiang Mai, Thailand to physically contribute to the construction of a school.
        \item Undertook training and assisted with a public astronomical observing night at McGill University.
    \end{itemize}
\end{itemize}


\section*{LANGUAGES}
\begin{itemize}
    \item English (fluent, language of education)

    \item Cantonese (fluent, mother tongue)

    \item Mandarin (formally learned 10 years)

    \item German (formally learned 10 years)
\end{itemize}

% \section*{CITIZENSHIPS}
% \begin{itemize}
%     \item United States of America (US)

%     \item Hong Kong SAR (HK)
% \end{itemize}




\section*{RECENT MEMBERSHIPS}
\begin{itemize}
    \item Sidney Sussex College Football Team, Premier Division

    \item Cambridge University Mountaineering Club

    \item Cambridge University Table Tennis Club

    \item Cambridge University Hong Kong Postgraduate Student Association

    \item Cambridge University Chinese Society
\end{itemize}

\section*{PERSONAL INTERESTS}
\begin{itemize}
    \item Rock climbing (both outdoors and indoors, with awards won at amateur, inter-high-school and inter-university competitions. Also led a rock climbing extra-curricular activity during high school, including captaining the competition team)

    \item Soccer (7+ years league participation)

    \item Scuba diving (PADI open water certified)

    \item Traveling
\end{itemize}

% \section*{OTHER CERTIFICATIONS}
% \begin{itemize}
%     \item Rock climbing level 1

%     \item German Deutsches Sprachdiplom

%     \item PADI Open Water Diver

%     \item ABRSM Piano Grade 8

%     \item ABRSM Music Theory Grade 5
% \end{itemize}


% End document
\end{document}
